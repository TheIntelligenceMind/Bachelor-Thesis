\section{Vorwort}

Im heutigen Zeitalter werden Informationssysteme, eingebettete Systeme und andere informationstechnische Systeme, die uns Menschen die Arbeit erleichtern oder sogar ganz abnehmen sollen, immer wichtiger und sind sogar größtenteils nicht mehr wegzudenken. Verschiedenste Berufe die noch vor Jahren ohne den Einsatz von solchen Systemen auskamen sind heutzutage gezwungen diese technologischen Neuheiten zu nutzen, um sich gegen Konkurrenten durchsetzen zu können. Computersysteme mit spezieller Software, eine eigene Webseite oder andere administrative Oberflächen zur Steuerung und Verwaltung bestimmter Komponenten, sind für viele Unternehmen essentiell wichtig. Mit dieser zunehmenden Vielfalt an Systemen steigt auch die Mensch-System-Interaktion und damit die Bedeutung von ergonomischen Benutzeroberflächen. Was vor ziemlich genau 50 Jahren noch die ersten Gehversuche mit ineffizienten Konzepten und verpixelten Grafiken waren, ist heute ein Zusammenspiel aus Form und Farb Kompositionen, Echtzeit Verarbeitung und der gesammelten UX der letzten Jahrzehnte als etablierter Standard für informationstechnische Systeme. Anwendungen mit grafischer Oberfläche definieren sich als aller erstes durch ihr Äußeres. Ist dies für den Anwender bzw. Endnutzer nicht zufriedenstellend hilft meist auch keine gute Performance oder ein überdurchschnittlicher Funktionsumfang mehr aus, damit sich der erste Eindruck rekapitulieren lässt. Unternehmen, Institute und Usability-Experten konnten in der Vergangenheit viele Erfahrungen sammeln und das Erlebnis, das sich User beim Gebrauch von Benutzeroberflächen wünschen, bewerten, analysieren und anschließend in Gestaltgesetze, Empfehlungen, Verordnungen und Richtlinien umformen. Mit zunehmenden technologischen Möglichkeiten wie zum Beispiel Smartphones, Tablets oder dem noch eher jungen Thema IoT\footnote{IoT (Internet of Things) = IT-Systeme die mittels Maschine-zu-Maschine-Kommunikation miteinander kommunizieren} werden auch die Ansprüche immer höher, denen eine gute Benutzeroberfläche gerecht werden muss. Das eine Webseite mit nahezu jedem Endgeräten kompatibel ist und ein responsives Designkonzept besitzt ist heutzutage gang und gäbe.

Zur erhöhten Benutzerfreundlichkeit und effizienteren und effektiveren Interaktion zwischen Mensch und Maschine wurden mit der Zeit mehrere Normen und Richtlinien entwickelt. Einer dieser Normen ist die Reihe DIN EN ISO 9241. Die neuste Auflage der Normreihe ist von 2007 und beinhaltet unter anderem Anforderungen an den Arbeitsplatz, Leitlinien zur Dialoggestaltung und Analyse-/Prüfverfahren für die optische Anzeige. Darüber hinaus gibt es noch weitere Richtlinien die Vorgaben zur Ergonomie am Arbeitsplatz beinhalten, so wie die Arbeitsstättenverordnung oder auch die Bildschirmarbeitsverordnung.

Persönlich hat mich schon seit Anfang an die Entwicklung, Gestaltung und Umsetzung von Anwendungen, speziell derer die ein grafisches Frontend\footnote{Frontend = Benutzeroberfläche} für Anwender besitzen, interessiert. Ich konnte mich schon immer dafür motivieren eigene Oberflächen zu gestalten und war bei der \gls{IFM} ebenfalls mit der Veränderung, Erweiterung oder Verbesserung von Benutzeroberflächen vertraut. Die Einblicke die ich bekam zeigten mir, dass der Fokus nie wirklich auf den Oberflächen der Software lag. Ich konnte mitverfolgen wie Oberflächen Gestalt, Sinn und Zweck verloren haben, nach dem sie jahrelang einfach nur historisch gewachsen sind ohne das sie ein Designkonzept besaßen. Meiner Ansicht nach wurde oftmals zu wenig Zeit für Oberflächen- und Steuerungskonzepte der Dialoge verwendet, wodurch Endresultate, gerade für den Anwender, nicht wirklich zufriedenstellend waren. Das Bestreben von Unternehmen, welche Softwarelösungen entwickeln, den ersten Kontakt mit einer neuen Software, genauer gesagt mit ihrer Oberfläche, positiv zu gestalten und das Arbeiten mit der \gls{GUI} so angenehm wie möglich zu machen, war einer meiner ersten Beweggründe die Ergonomie von grafischen Benutzeroberflächen in meiner Ausarbeitung zu betrachten. 

Die vorliegende Ausarbeitung soll über Ziele, Ist-Situation, Grundlagen, Umsetzungen, Auswertungen bis hin zu den Ergebnissen eine Einführung in die Thematik der Benutzerfreundlichkeit von Oberflächen geben und zum anderen die angewandten Methodiken und die erzielten Resultate bewerten. Dabei werden immer wieder Beispiele als Hilfestellung für Sachverhalte und Methoden eingebracht. Dabei ist darauf hinzuweisen, dass jegliche Erklärungen und Beispiele im generischen Maskulinum formuliert sind. Das heißt z.B. das nur der Begriff Teilnehmer anstatt Teilnehmerin oder beider Formen verwendet wird. Jedoch ist explizit gesagt, dass trotzdem beide Geschlechter gemeint sind. Der Einsatz des generischen Maskulinums wird lediglich durch seine kürzere Schreibweise und Einfachheit begründet.
