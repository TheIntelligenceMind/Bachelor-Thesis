\section{Ergebnisse und Diskussion}
\label{sec:ergebnisseUndDiskussion}
Im nachfolgenden Kapitel werden im ersten Schritt die Ergebnisse der Datenerhebungsmaßnahmen ausgewertet und graphisch aufbereitet. Im zweiten Schritt sollen die aufbereiteten Daten interpretiert, diskutiert und miteinander in Bezug gesetzt werden. Zusätzlich soll eine reduzierte Form der \gls{KNA} durchgeführt werden, um eine Einschätzung über die Wirtschaftlichkeit des Projekts zu erhalten. Als letzten Schritt werden die Hypothesen sowie die Forschungsfrage erneut aufgegriffen und mit Hilfe der gewonnen Erkenntnisse überprüft und beantwortet.  

%%%%%%%%%%%%%%%%%%%%%%%%%%%%%%%%%%%%%%%%%%%%%%%%%%%%%%%%%%%%%%%%%%%%%%%%%%%%

\subsection{Auswertung der Logfiles}
\label{sec:auswertungDerLogfiles}
Die Daten der erhobenen Logfiles befinden sich im Ausgangszustand in einer Oracle Datenbank und können von da aus weiter verarbeitet werden. Um einen ersten Eindruck über die Daten zu bekommen, wurden bereits während der Erhebungen Datenbankabfragen entwickelt mit denen Datensätze aggregiert werden können. Dadurch konnte frühzeitig sichergestellt werden, dass die Daten ohne Probleme und in zufriedenstellender Form gesammelt und gespeichert werden. Unversehrte Daten sind für die nachfolgenden Auswertungsschritte die wichtigste Voraussetzung.

Für genauere Auswertungen wurde die Beobachtung der Logfile-Daten mit Hinblick auf die Ziele der Arbeit fokussiert. Zu Anfang soll sich bewusst erst auf die aufgestellten Kriterien und Metriken bezogen werden. In einem zweiten Schritt sollen Auffälligkeiten genauer analysiert werden und mögliche Zusammenhänge hinterfragt werden.

\textbf{Bearbeitungszeiten}

Einer der Kriterien, die im Zuge des direkten Vergleichs zwischen dem alten und neuen Dialog aufgegriffen werden sollen, ist die Bearbeitungszeit. Die Bearbeitungszeit ist die Zeit, die ein Sachbearbeiter benötigt, um ein Vermögensverzeichnis in die Cosima Oberfläche einzugeben und anschließend erfolgreich zu speichern. Bedeutet auf der anderen Seite, dass Vorgänge bei denen Sachbearbeiter den Dialog nach Eingabe von Daten ohne zu speichern schließen oder den Dialog öffnen, jedoch keine Daten eingeben, nicht dazu zählen. Damit eine Aussage über die Bearbeitungszeit getroffen werden kann, müssen vergleichbare Daten zugrunde liegen. Dafür benötigt es eine geeignete Aggregation der Datensätze.

Der erste Schritt der Datenaggregation sieht vor, nach Art des Dialogs (alt oder neu) zu unterscheiden. Dafür dient das am Session-Objekt hängende Attribut \enquote{DialogID}. Als zweiter Schritt werden die Sessions herausgefiltert, die eine Speicherung der Daten beinhalten. Dazu sollen auch nur die Sessions genutzt werden bei denen mindestens eine Veränderung der Daten stattgefunden hat. Bloßes Öffnen des Dialogs zum Einsehen von Informationen wird nicht gewertet. Für die Einhaltung der aufgestellten Bedingungen werden die beiden Attribute \enquote{WertGeaendertFlag} und \enquote{GespeichertFlag} genutzt.

Alle Sessions, die den oben aufgestellten Bedingungen entsprechen, gehören zu den gesuchten Bearbeitungsvorgängen. Die Vorgänge werden in einem weiteren Schritt auf einzelne Tage gruppiert. Es lässt sich bereits erkennen, dass die Menge der Bearbeitungsvorgänge mit dem alten Dialog (752 Sessions), nur 15 Sessions mehr beinhaltet als die Menge der erhobenen Bearbeitungsvorgänge mit dem neuen Dialog (737 Sessions). Für einen nachfolgend aussagekräftigen Vergleich von Durchschnittswerten, die auf Basis dieser Mengen berechnet werden, sind die beieinander liegende Stichprobenumfänge eine gute Ausgangslage.

Mit Hilfe der Sessions-Attribute \enquote{Startzeitpunkt} und \enquote{Endzeitpunkt} können die Bearbeitungszeiten einzelner Sessions ermittelt werden und anschließend zusammengeschlossen werden, um den Mittelwert auf Tagesbasis zu berechnen. Da die resultierenden Daten vereinzelt Spitzen in den Bearbeitungszeiten aufzeigen, wird der gleitende Durchschnitt (GD) dritter Ordnung angewandt, um den Verlauf zu glätten. Ursprünglich wurden beide Dialoge jeweils über einen Zeitraum von 11 Tagen gemessen, jedoch fallen nun die ersten beiden Periodenwerte aufgrund des GD weg. Es bleiben neun Perioden für die Darstellung des zeitlichen Verlaufs der Bearbeitungszeiten. Diese sollten ausreichen, um einen Eindruck über die Entwicklung zu erhalten.

Werden nun die Bearbeitungszeit (Sekunden) gegen die Periode (Tage) in einem Diagramm aufgetragen sieht der geglättete Verlauf der durchschnittlichen Bearbeitungszeiten wie folgt aus:
\begin{figure}[H]
  \centering
  \includegraphics[]{img/Bearbeitungszeit_Verlauf.png}
  \caption{Vergleich des Verlaufs der durchschnittlichen Bearbeitungszeiten beider Dialoge.}
  \caption*{\textbf{Quelle:} Eigene Darstellung}
  \label{fig:BearbeitungszeitVerlauf}
\end{figure}
Es lässt sich erkennen, dass die durchschnittliche Bearbeitungszeit pro Session im alten Dialog (rote Linie) im Bereich von 140 Sekunden bis 160 Sekunden leicht schwankt. Zudem ist die gemittelte Bearbeitungszeit über alle Perioden mit 147 Sekunden, 14 Sekunden über der des neuen Dialogs (133 Sekunden).

Eine weitere Auffälligkeit innerhalb des Verlaufs des neuen Dialogs sind die ersten vier Perioden. In diesen sinkt die Bearbeitungszeit kontinuierlich von etwa 160 Sekunden auf durchschnittlich 120 Sekunden ab. Ab dort ist der Verlauf bis zum Ende der siebten Periode konstant. Anschließend steigt die Bearbeitungszeit wieder leicht auf 135 Sekunden an. Hingegen ist zu erkennen, dass die blaue Linie ab der zweiten Periode durchgehend unterhalb der roten Linie verläuft. Also die durchschnittliche Bearbeitungszeit im neuen Dialog, fast durchgängig geringer ist, als die im alten Dialog. Diese Tatsache sollte aber mit gewisser Vorsichtig betrachtet werden, da die Perioden, also sowohl der Wochentag als auch der anfallende Arbeitsaufwand an dem Tag, in denen die Erhebungen durchgeführt wurden nicht identisch sind.

\textbf{Mausklicks und Tastaturanschläge}

Als weitere Kennzahl für den Vergleich der Dialoge, dient die Anzahl von Interaktionen insgesamt, sowie deren Unterteilung in Mausklicks und Tastaturanschläge. Hiermit sollen Hinweise gesammelt werden, die dabei helfen können eine Aussage über die Komplexität und die physische Belastung bei der Arbeit mit den Dialogen zu treffen.

Als Erstes sollen die oben bereits ermittelten Bearbeitungsvorgänge genutzt werden, um alle Interaktionen zu bekommen. Es lässt sich anschließend mit Hilfe des Interaktions-Attributs \enquote{Typ} eine Unterscheidung zwischen Maus und Tastatur treffen. In Abbildung \ref{fig:verlaufInteraktionenAlterDialog} sowie in Abbildung \ref{fig:verlaufInteraktionenNeuerDialog} wurden die Interaktionen nach Typ gestaffelt gegen die Perioden in Tagen in einem Balkendiagramm aufgetragen. Die Datentabellen mit allen Werten zu den Interaktionen beider Dialoge ist im Anhang \ref{sec:datentabelleAlterDialog} bzw. \ref{sec:datentabelleNeuerDialog}zu finden. Es ist darauf zu achten, dass Interaktionen nur in ganzzahligen Werten in der Realität existieren. Die dargestellten Mittelwerte sind jedoch Dezimalzahlen mit zwei Nachkommastellen, um die Verhältnisse genauer darzustellen.
\begin{figure}[H]
  \centering
  \includegraphics[]{img/Interaktionen_Alter_Dialog.png}
  \caption{Verlauf der Interaktionen im alten Dialog.}
  \caption*{\textbf{Quelle:} Eigene Darstellung}
  \label{fig:verlaufInteraktionenAlterDialog}
\end{figure}
Werden die Interaktionen pro Periode im alten Dialog betrachtet, lässt sich erkennen, dass die Werte pro Periode gerundet zwischen 18,4 und 24,7 Interaktionen schwanken. Werden Tastatur- und Maus-Interaktionen relativ zueinander erfasst, so wird die Tastatur zwischen 22,95\% und 28,42\% im Durchschnitt zu 25,06\% genutzt. Die Maus hingegen besitzt ein Maximum von 77,05\% und ein Minimum von 71,58\% an den gesamten Interaktionen einer Periode. Im Mittel wird die Maus zu 74,94\% verwendet.

\begin{figure}[H]
  \centering
  \includegraphics[]{img/Interaktionen_Neuer_Dialog.png}
  \caption{Verlauf der Interaktionen im neuen Dialog.}
  \caption*{\textbf{Quelle:} Eigene Darstellung}
  \label{fig:verlaufInteraktionenNeuerDialog}
\end{figure}
Im neuen Dialog ist zu erkennen, dass die Werte im Schnitt zwischen 14,5 und 19 Interaktionen liegen. Bei der Ermittlung von relativen Anteilen, liegt der Anteil von Tastatur-Interaktionen in einem Bereich von 41,67\% bis 49,67\%. Die anteilige Benutzung der Maus liegt zwischen 50,33\% und 58,33\%. Im Durchschnitt wird die Tastatur für 44,86\% und die Maus für 55,14\% der Interaktionen genutzt.

% Vergleich
Werden beide Dialoge miteinander verglichen so ist zum einen festzustellen, dass die Interaktionen im neuen Dialog geringere Schwankungen verzeichnen und zum anderen ist die durchschnittliche Anzahl von Interaktionen pro Session gemittelt über alle Perioden im alten Dialog etwa 25\% höher als im neuen Dialog. Zum anderen hat sich das Verhältnis von Tastatur- und Maus-Interaktionen verschoben. Mit dem neuen Dialog sind die Tastatur-Interaktionen um knapp 20\% gestiegen, dafür ist die Benutzung der Maus um 20\% zurückgegangen. Insgesamt wurden im neuen Dialog durchschnittlich pro Session knapp 4 Interaktionen weniger ausgeübt, als im alten Dialog.

\textbf{Anzahl und relative Häufigkeit von Veränderungen}

Wie bereits oben genannt wurden die Daten in zwei hintereinander liegenden Zeiträumen erhoben, in denen je nach aktuellem Aufkommen unterschiedlich viele Vorgänge mit unterschiedlichen Komplexitäten vorliegen. Die Komplexität soll im folgenden mit der Anzahl an Veränderungen bzw. den Eingaben von Daten auf der Oberfläche gleichgesetzt werden. Das heißt, umso mehr Daten einzugeben sind, desto komplexer ist die Bearbeitung des Vorgangs zu bewerten. Die Komplexität soll dabei helfen beide Stichprobenumfänge die Datengrundlage beider Dialog besser miteinander vergleichen zu können.

Um die Komplexität zu vergleichen wurde die relative Häufigkeit von Veränderungen gegen die Anzahl von Veränderungen (bis 50 Veränderungen) pro Session in einem Balkendiagramm (siehe Anhang \ref{sec:verteilungVeraenderungen}) aufgetragen. Aufgrund der Größe der Grafik wurde diese bewusst in den Anhang der Ausarbeitung verschoben. 

Der Komplexitätsvergleich wird auf der Grundlage von 777 Sessions aus dem neuen Dialog und 781 aus dem alten Dialog aufgestellt. Es fällt auf, dass die Verteilung von wenigen Veränderungen zu vielen Veränderungen innerhalb einer Session (gemessen am GD zweiter Ordnung) bei beiden Dialogen unterschiedlich aussieht. Der neue Dialog hat einen größeren Anteil von weniger komplexen Sessions (5 bis 13 Veränderungen). Im alten Dialog wiederum ist der Großteil der Sessions im Bereich von 11 bis 22 Veränderungen angesiedelt. Ab 35 Veränderungen ähneln sich beide Graphen wieder relativ stark. 


\textbf{Weitere Auffälligkeiten und Zusammenhänge}
%%%%%%%%%%%%%%%%%%%%%%%%%%%%%%%%%%%%%%%%%%%%%%%%%%%%%%%%%%%%%%%%%%%%%%%%%%%%

\subsection{Auswertung der Fragebögen}
\label{sec:auswertungDerFrageboegen}
Für die Auswertung der Fragebögen, die im Kapitel \ref{sec:durchfuehrungEvaluation} von den Probanden erhoben wurden, werden alle Datensätze in eine Excel-Tabelle übertragen. Daraufhin werden die Daten so angeordnet, dass jede Zeile einem Bewertungsaspekt aus dem ISO Fragebogen entspricht (siehe Abb. \ref{fig:auswertungsmatrixNeuerDialog} und Anhang \ref{fig:auswertungsmatrixAlterDialog}). Die Ausprägungen für die Bewertung eines Items werden so formatiert, dass sie anstatt von \enquote{$---$} bis \enquote{$+++$} den Werten von -3 (sehr negativ) bis +3 (sehr positiv) entsprechen. Dies hat den Vorteil das anschließend mit den numerischen Werten weiter gerechnet werden kann, da sie ein repräsentatives Abbild der Antworten darstellen und der Ordnung der Likert-Skala entsprechen\footnote{\cite[vgl.][]{Statista}}. Zudem wurden die Kennzahlen in den Auswertungsmatrizen farblich hinterlegt, damit zwischen negativen (rot) und positiven (grün) Bewertungen unterschieden werden kann. Das Rating der 35 Bewertungsaspekte wurde mit Hilfe der Mittelwerte aus den Bewertungen aller Teilnehmer berechnet. Die Benutzerzufriedenheit eines Gestaltungsgrundsatz wird ebenfalls durch den Mittelwert der fünf zuvor berechneten Mittelwerte der Bewertungsaspekte gebildet und besitzt anschließend einen Wert zwischen -3 und +3. Aus Darstellungsgründen wurden die Zahlen im Folgenden alle auf drei Stellen nach dem Komma gerundet.

In Abbildung \ref{fig:auswertungsmatrixNeuerDialog} sind alle Einschätzungen zu dem neuen Vermögensverzeichnis Dialog in aggregierter Form dargestellt. Stellt man diese Auswertungsmatrix in einen direkten Vergleich zu den Auswertungen des alten Dialogs (siehe Abb. \ref{fig:auswertungsmatrixAlterDialog}) lässt sich erkennen, dass der neue Dialog in allen sieben Gestaltungsgrundsätzen besser bewertet wurde, als der alte Dialog (siehe Abb. \ref{fig:vergleichBalkendiagramm}). 
\begin{figure}[H]
  \centering
  \includegraphics[]{img/ISO9241-10_Vergleich_Balkendiagramm.PNG}
  \caption{Vergleich der Gestaltungsgrundsätze zwischen altem und neuen Dialog.}
  \caption*{\textbf{Quelle:} Eigene Darstellung}
  \label{fig:vergleichBalkendiagramm}
\end{figure}
Die größte Steigerung kann in den Gebieten Aufgabenangemessenheit und Lernförderlichkeit mit 1,1 Punkten und 1,4 Punkten verzeichnet werden. Am wenigsten konnte die Benutzerzufriedenheit bei der Steuerbarkeit und der Individualisierbarkeit steigen. 

Trotz der insgesamt positiven Ergebnisse, gibt es immer noch Bereiche in denen die Bewertungen eher schwach ausfallen. Dazu zählt auch die Steuerbarkeit des Dialogs. Diese ist zwar positiv, also bietet ausreichend Freiheit bei der Bedienung, jedoch empfinden die Benutzer die Unterbrechungsfreiheit und die Anpassbarkeit von Informationsdarstellungen noch nicht vollkommen zufriedenstellend. Dieses Thema ist eher komplex zu bewerten, da hier auf die einzelnen Wünsche und Belangen der einzelnen Sachbearbeiter eingegangen werden müsste, um weitere Verbesserungen zu erzielen.

Auch die Kategorie Individualisierbarkeit selbst konnte im direkten Vergleich einen Gewinn von knapp 0,7 Punkte verzeichnen. Jedoch liegt die Bewertung mit +0,219 im unteren positiven Bereich. Besonders die Punkte \enquote{Erweiterbarkeit durch den Benutzer} und \enquote{Individuelle Anpassbarkeit durch den Benutzer} fallen beide noch schlecht aus. Gleichzeitig sind die beiden Bewertungsaspekte, trotz ihres Anstiegs um 0,4 bzw. 0,6 Punkten, auch die einzigen die noch negativ bewertet wurden. Sie spiegeln die individuelle Erweiterbarkeit und Anpassbarkeit von Bearbeitungsschritte und Benutzer wider. Die Kritik an dieser Stelle ist aber durchaus nachzuvollziehen, da sich der Dialog nur an wenigen Stellen individuell anpassen lässt. Ein Grund dafür ist die niedrige Priorisierung von individuellen und dynamischen Strukturen während der Konzeptionsphase. Es ist denkbar in einem nachfolgenden Projekt noch einmal explizit auf die beiden Aspekte und Benutzeranliegen einzugehen, um bestenfalls die Benutzerzufriedenheit weiter zu verbessern.

Insgesamt ist die Selbstbeschreibungsfähigkeit mit einem Wert von +1,067 durchaus zufriedenstellend. In Bezug auf die, durch den Benutzer geforderten und die durch das System automatisierten Erklärungen, ist aber noch klar ein Verbesserungspotential zu erkennen. Hier könnten gezielte Maßnahmen eingeleitet werden, bei denen zusammen mit dem Fachbereich fehlende Erklärungen evaluiert und anschließend ergänzt werden. Dies sollte einen vergleichsweise geringen Aufwand darstellen und kann auf der anderen Seite große Effekte erzielen.

Mit dem Blick auf die positiven Eigenschaften des neuen Vermögensverzeichnis Dialogs fallen unter die besten fünf Punkte:
\begin{enumerate}
    \item Hilfe beim Behalten von Gelerntem (+2,104),
    \item Autodidaktisch\footnote{Autodidaktisch = Wissen durch Literatur, Übungen, Beobachtungen oder Versuche eigenständig aneignen } (+2,085),
    \item Bedienungskomplexität (+2,083),
    \item Zeit für Einarbeitung (+2,063) und
    \item Erfordernis, sich Details zu merken (+1,979).
\end{enumerate}
Interessant ist, dass vier der fünf Punkte aus dem Bereich Lernförderlichkeit stammen. Der Aspekt \enquote{Bedienungskomplexität} wiederum gehört zum Grundsatz Aufgabenangemessenheit. Hier bewerten die Benutzer die Bedienung des neuen Dialogs, mit einem durchschnittlichen Schwellwert von etwa +2, als angenehm und einfach. Im Vergleich ist dies ein Anstieg von über 1,6 Punkten. Die gestiegene Benutzerzufriedenheit von über 1,4 Punkten im Bereich der Lernförderlichkeit kann genauso wie die Bedienungskomplexität durch die Vereinfachung und Verringerung von Informationen, Bedienelementen und Strukturen erklärt werden. 
\begin{figure}[H]
  \centering
  \includegraphics[width=430px]{img/Auswertungsmatrix_Neuer_Dialog.PNG}
  \caption{Auswertungsmatrix zum ISO 9241-10 Fragebogen neuer Dialog.}
  \caption*{\textbf{Quelle:} Eigene Darstellung}
  \label{fig:auswertungsmatrixNeuerDialog}
\end{figure}
Die fünf Bewertungsaspekte sollten gerade mit Blick auf die Zeit positive Auswirkungen haben. Fällt die Betrachtung auf die Punkte zwei und vier so lässt sich ableiten, dass potentiell künftig Zeit eingespart werden kann. Es muss sich lediglich der Anwendungsfall \enquote{Neuer Mitarbeiter} vorgestellt werden. Der neue Mitarbeiter benötigt tendenziell weniger Zeit für die Einarbeitung in den neuen Dialog. Zudem kann der zeitliche Aufwand für helfende und einweisende Kollegen verringert werden.

So lassen sich auch die Punkte \enquote{Erfordernis, sich Details zu merken} und \enquote{Hilfe beim Behalten von Gelerntem} mit einer gesteigerten Effizienz in Verbindung bringen. Ein möglicher Anwendungsfall hierfür ist ein Vermögensverzeichnis, das viele Details enthält. Hier muss der Sachbearbeiter im Gegensatz zum alten Dialog nur noch einen Bruchteil der Informationen übernehmen und somit auch weniger Informationen gleichzeitig im Gedächtnis behalten. Dies kann durchaus positive Effekte auf die Bearbeitungszeit und die mentale Belastung des Sachbearbeiters nehmen.

Im Gesamtüberblick kann bereits ein erfreulicher Verlauf der Benutzerzufriedenheit in allen sieben Grundsätzen verzeichnet werden. Hieraus lässt sich bereits auf eine gewisse Effektivität des Projekts schließen, besonders mit Bezug auf die zeitlichen Effekte. Die Benutzer scheinen allgemein zufriedener bei der Arbeit mit dem neuen Dialog.

%%%%%%%%%%%%%%%%%%%%%%%%%%%%%%%%%%%%%%%%%%%%%%%%%%%%%%%%%%%%%%%%%%%%%%%%%%%%

\subsection{Abschließende Ergebnisdiskussion}
\label{sec:abschliessendeErgebnisdiskussion}
Im letzten Schritt werden die gesammelten Erkenntnisse zusammengefasst und in einen Zusammenhang gebracht. Zusätzlich werden die Ergebnisse mit weiteren Argumenten und verschiedenen Blickrichtungen angereichert, um abschließend die drei Hypothesen H1\textsubscript{0}, H2\textsubscript{0} und H3\textsubscript{0} zu belegen bzw. zu widerlegen.

Hinsichtlich der ersten Hypothese H1\textsubscript{0} ist mit den vorliegenden Fakten und Annahmen zu überprüfen ob diese belegt werden kann oder nicht. 