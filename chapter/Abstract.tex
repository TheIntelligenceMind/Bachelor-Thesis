\section{Abstract}
Die vorliegende Bachelorarbeit befasst sich mit der Ergonomie von Benutzeroberflächen. Dabei ist es das primäre Ziel, die Oberfläche für Vermögensauskünfte, als Teil der IFM Inkassosoftware, zu erneuern. Als Ausgangslage für die Ausarbeitung, wird die Forschungsfrage: \enquote{Weist der neue Eingabedialog für Vermögensverzeichnisse eine höhere Gebrauchstauglichkeit auf als der alte Dialog?}, aufgestellt.

Für die Erneuerung der Oberfläche wird sich an verschiedenen Methoden und Techniken aus den Bereichen Vorgehensmodelle der Entwicklung, Grundsätze und Prinzipien der Gestaltung von Benutzeroberflächen, Evaluationsmethoden und statistische Auswertung von Daten bedient. Als Vorgehensmodell wird sich für das agile Vorgehen des Usability Engineerings entschieden. Zudem wird mit Hilfe von Gestaltungsgrundsätzen, Normen, Prinzipien und Gesetzen ein theoretischer Ansatz für die Gestaltung einer ergonomischen Benutzeroberfläche geschaffen. Bei der anschließenden Evaluation ist es das Ziel, die neue Oberfläche mit der alten Oberfläche zu vergleichen. Hierfür wurden sowohl Befragungen als auch Logfiles erhoben. Die abschließende Auswertung der Daten führte zu dem Ergebnis, dass eine Verbesserung in der Effizienz und Zufriedenheit der neuen Benutzeroberfläche erreicht werden konnte. Zudem wurde mit 52 eingesparten Arbeitstagen ein Richtwert für den jährlich erzielten Nutzen des Projekts ermittelt.