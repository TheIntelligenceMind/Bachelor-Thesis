%!TEX root = ../Thesis.tex
\section{Einleitung}


\subsection{Unternehmensvorstellung}
\subsubsection{Bertelsmann SE \& Co. KGaA} Die Bertelsmann SE \& Co. KGaA (im folgenden nur noch als Bertelsmann bezeichnet) ist ein international erfolgreicher IT-Dienstleistungskonzern. Der Hauptsitz liegt in Gütersloh und geht zurück auf den Drucker und Buchbinder Carl Bertelsmann, der 1835 den damaligen Buchverlag Bertelsmann gründete. Inzwischen hat sich der Verlag zu einem weltweit tätigen Medien-, Dienstleistungs- und Bildungsunternehmen, unter der Führung des Vorstandsvorsitzenden Thomas Rabe und dem Aufsichtsratvorsitzenden Christoph Mohn, weiterentwickelt. Zu den wichtigsten Geschäftsbereichen und Tochtergesellschaften zählen die RTL Group, Penguin Random House, Gruner + Jahr, BMG, Arvato, die Bertelsmann Printing Group, die Bertelsmann Education Group und Bertelsmann Investments. Die Tochtergesellschaft Arvato, die zu 100\% zu Bertelsmann gehört, bietet beispielsweise Services und Lösungen im Bereich Finanzen, CRM, SCM und IT für Unternehmen rund um den Globus an \citep{BertelsmannGeschaeftsbericht2016}.

\subsubsection{Arvato infoscore Forderungsmanagement GmbH}
Einer der Tochtergesellschaften im Bereich der Finanzdienstleistungen ist die Arvato infoscore Forderungsmanagement GmbH (IFM).

\subsection{Ist-Situation und Problemstellung}
Die IFM setzt eine eigens entwickelte Softwarelösung Namens COSIMA ein, die in verschiedensten Bereichen des Unternehmens Anwendung findet. Die Kernanwendung basiert auf einer Client-Server-Architektur und ist größtenteils in Java programmiert. COSIMA ist für das Verwalten von wichtigen Daten, die im Laufe des Inkassoprozesses aufkommen zuständig und wird von Mitarbeitern aus den unterschiedlichen Fachbereichen und \gls{SDC}'s bei der täglichen Arbeit genutzt.

Durch zusätzliche Features, die die Sachbearbeiter für ihre Arbeit benötigen, ist die Software mit der Zeit immer weiter gewachsen. Meist bestehen solche Features aus einer erweiterten Datenstruktur und letztendlich einer Veränderung, Anpassung beziehungsweise Erweiterung von Benutzerschnittstellen. COSIMA besitzt über hundert verschiedene Dialoge zur Anzeige und Bearbeitung von Daten. Einer dieser Dialoge ist für die Verwaltung von Schuldnerinformationen zuständig. Genauer gesagt werden dort Informationen zur Lebenssituation, Finanzsituation und dem Vermögen eines Schuldners gespeichert. Die Oberfläche wird mittlerweile für mehr als nur einen Anwendungsfall eingesetzt und besitzt daher einige Bedien- und Eingabeelemente, die für den ursprünglichen Use-Case unrelevant sind. Dadurch das der Dialog für mehrere Anwendungsfälle verwendet wird aber nicht jeder Anwendungsfall die kompletten Daten aus dem Dialog benötigt, wird ein Overhead\footnote{Overhead = Daten die nicht zu den primären Nutzdaten zählen} erzeugt, der das Arbeiten mit dem Dialog unnötig verkompliziert. Sachbearbeiter finden den Dialog inzwischen sehr unübersichtlich und die Benutzung eher schwerfällig und ineffizient.

Einer dieser Anwendungsfälle tritt immer dann auf, wenn Schuldner ihre ausstehenden Forderungen nicht begleichen möchten oder nach eigenen Angaben nicht können. In diesen Fällen wird dann eine Zwangsvollstreckung eingeleitet. Dazu wird der Schuldner von einem, durch die IFM beauftragten Gerichtsvollzieher, heimgesucht, um einen Vermögensantrag zu seiner aktuellen Lebens- und Finanzsituation auszufüllen. Dieser Antrag besitzt einen standardisierten Aufbau und wird nach dem er vom Schuldner ausgefüllt wurde an die IFM geschickt. Als nächstes wird durch das System automatisch ein Listeneintrag\footnote{Listeneintrag = ein Arbeitsauftrag der für Sachbearbeiter im System erscheint} erstellt, der einem zuständigen Sachbearbeiter zugeteilt wird. Im ersten Arbeitsschritt öffnet der Mitarbeiter das, als PDF vorliegende, Dokument auf einem seiner Monitore. Gleichzeitig öffnet er den Dialog für Schuldnerinformationen auf einem weiteren Monitor. Nach und Nach werden dann die vorliegenden Angaben aus dem Vermögensantrag in den Dialog übernommen und am Ende abgespeichert. 

Das Problem speziell in diesem Fall liegt darin, dass der Sachbearbeiter für zwei untereinander folgenden Informationen aus dem Antrag teilweise an mehrere verschiedenen Stellen in dem Dialog springen muss, um diese zu übernehmen. Der Arbeitsablauf wird durch solche Sprünge unterbrochen und der Bearbeiter muss sich häufig neu orientieren. Zudem ist der Dialog mit zunehmender Anzahl an Bedienelementen, auch immer unübersichtlicher geworden. Sachbearbeiter berichten immer wieder von einer zu komplexen und für sie nicht effizienten Benutzeroberfläche. Grundsätzlich brauchen auch neue Arbeitskräfte bei der Einarbeitung mit diesem Dialog länger als gewünscht. Aus diesen Gründen sind sich die Beteiligten einig das eine Veränderung durchgeführt werden muss. Der aktuelle Dialog ist in die Jahre gekommen und entspricht nicht mehr den Anforderungen der Verantwortlichen.



\subsection{Zielsetzung, Forschungsfrage und Hypothesen}
Die Ist-Situation inklusive ihrer Problemstellung, dient als Ansatzpunkt für mein Projektziel, das sich in zwei Unterziele aufteilt. Der erste Teil des Ziels besteht aus der Konzeption und Implementierung eines neuen Dialogs für die Eingabe von Vermögensverzeichnissen. Im zweiten Schritt soll dieser Dialog mit dem bestehenden Dialog für Vermögensverzeichnisse auf Basis einer empirischen Datenerhebung in den Disziplinen Ergonomie, Effizienz und Performance verglichen werden. Die zu erhebenden Daten bestehen ebenfalls aus zwei Teilen. Zum einen den objektiv technisch erhobenen Daten und zum anderen den mehr subjektiv und aus einem Fragebogen entstammenden Daten. Am Ende soll ein neuer Dialog zur Verfügung stehen, der dem Sachbearbeiter eine ergonomische, handlungsorientierte und performante Eingabe von Vermögensverzeichnissen ermöglicht. Dabei soll dieses Projekt der erste Schritt für die Ablösung des alten Dialoges dienen. Das Projekt startet gleichzeitig als Pilotprojekt für den Vergleich von Dialogen auf Basis realer Daten und dient somit als Referenzprojekt für zukünftige Projekte im Bereich der Software-Ergonomie.

Im Verlaufe der Abschlussarbeit gilt es die zu erreichenden Ziele auch hinsichtlich aufgestellter Hypothesen und Antithesen zu überprüfen. Neben den Thesen gibt es eine zentrale Forschungsfrage, die mit den gewonnen Erkenntnissen und Ergebnissen am Ende der Thesis beantwortet werden soll. Die Hauptfrage mit der sich beschäftigt werden soll lautet: \glqq Ist der neue Dialog für Vermögensverzeichnisse ergonomischer und effizienter als der bisherige Dialog?\grqq{}. Aus dieser Fragestellung leiten sich weitere Unterfragen ab: \glqq Wird der neue Dialog als übersichtlicher und einfacher zu bedienen wahrgenommen?\grqq{} und \glqq Führt der neue Dialog zu geringeren Bearbeitungszeiten bei der Anlage von Vermögensverzeichnissen?\grqq{}.

Der Forschungsteil und seine genannte Forschungsfrage wird versucht mit Hilfe von verifizierten Hypothesen zu beantworten. Eine der zu verifizierenden Hypothesen in diesem Kontext heißt: \glqq Wenn der neue Dialog anstatt des alten Dialoges verwendet wird, dann sinkt die durchschnittliche Bearbeitungszeit bei der Anlage von Vermögensverzeichnissen\grqq{}. Eine weitere damit verbundene Hypothese lautet: \glqq Je weniger Bedienelemente eine Oberfläche besitzt desto weniger Interaktionen finden durch den Sachbearbeiter statt\grqq{}. Zudem werden diese Hypothesen ebenfalls als zu untersuchenden Antithesen umformuliert: \glqq Wenn der neue Dialog anstatt des alten Dialoges verwendet wird, dann stagniert oder steigt die durchschnittliche Bearbeitungszeit bei der Anlage von Vermögensverzeichnissen\grqq{}, \glqq Je weniger Bedienelemente eine Oberfläche besitzt, desto weniger Interaktionen finden durch den Sachbearbeiter statt\grqq{}


\subsection{Aufbau und Vorgehen}
Die Ausarbeitung gliedert sich in sieben Kapitel auf, von denen die ersten beiden Kapitel keinen inhaltlichen Mehrwert schaffen sondern die Motivation dieser Arbeit aufzeigen und eine grobe Zusammenfassung über alle Teile dieser Thesis geben.

Der erste Hauptabschnitt mit dem in das Thema dieser Arbeit eingeführt wird ist die \glqq Einleitung\glqq{}. In dem Teil wird die aktuelle Situation zu Beginn des Projektes und das bestehende Problem beschrieben. Darauf folgen dann die Zielsetzung, Forschungsfragen und Hypothesen die hinsichtlich der Fragestellungen zu überprüfen und zu belegen oder gegeben falls zu widerlegen sind. Mit dieser Sektion \glqq Aufbau und Vorgehen\grqq{} wird die Struktur und das Vorgehen der Ausarbeitung skizziert. In dem letzten Teil der Einleitung wird der Konzern Bertelsmann SE \& Co. KGaA und die Tochtergesellschaft arvato infoscore Forderungsmanagement GmbH, in der das Projekt durchgeführt wird, vorgestellt.

In der zweiten Sektion geht es darum die benötigten Grundlagen zu erklären, damit das Thema und umliegende Themengebiete der grafischen Oberflächen für jeden verständlich sind. Dafür wird zum einen auf die eingesetzten Technologien in Hinsicht auf die Oberflächentechnologie Swing eingegangen und auf weitere Komponenten die für die Projektdurchführung benötigt wurden. Darauf folgt dann der thematische Einstieg in die Softwareergonomie zu der einige Begriffe definiert, Gesetze, Richtlinien und Normen vorgestellt und theoretische Ansätze zum Designen von Benutzeroberflächen gezeigt werden. Danach werden mehrere Erhebungsmethoden exemplarisch herausgesucht und erklärt welchen Nutzen sie haben und welche Unterschiede zu anderen Methoden bestehen. Ein weiterer sehr wichtiger Teil sind die statistischen Methoden die im Anschluss erläutert werden. Es werden sowohl einige Grundbegriffe der Statistik erklärt als auch mehrere Methodiken zur Auswertung von Datenstämmen gezeigt. Der letzte Teil der Grundlagen besteht aus der Vorstellung der Nutzwertanalyse, die zum Ende des Projektes durchgeführt werden soll.

In Kapitel 5 \glqq Entwicklung und Umsetzung eines Tools zum Aufzeichnen von Bewegungsdaten\grqq{} geht es darum die Anforderungen, Funktionen und wichtigen Arbeitsschritte bei der Umsetzung eines Programms, das für das Sammeln und Speichern von Mensch-System-Interaktionen zuständig ist, zu beschreiben.

Der nächste Abschnitt behandelt die \glqq Entwicklung und Umsetzung der neuen Benutzerschnittstelle\grqq{} für die Eingabe von Vermögensanträgen. Dazu gehört als erstes die funktionalen und die nicht funktionalen Anforderungen an die Oberfläche zu klären und erste Abgrenzungen für die spätere Umsetzung vorzunehmen. Das Konzept unter Punkt 5.2 beschreibt die Frage \glqq Was\grqq{} wird mit der neuen Oberfläche verfolgt und unter Punkt 5.3 wird das Design der neuen Schnittstelle mittels zuvor vorgestellten Methoden und Ansätzen erarbeitet. Anschließend folgt die Realisierung, dass heißt es werden alle benötigten Programmier- und Entwicklertests werden auf einer Entwicklungsumgebung durchgeführt. Nach Beendigung der Entwicklungsphase wird der neue Dialog durch Tester mit Hilfe von Integrationstest auf einer Abnahmeumgebung auf ihre Funktionalitäten getestet.

In dem Kapitel \glqq Empirische Erhebung\grqq{} wird anhand einer ausgesuchten Gruppe von Sachbearbeitern eine Erhebung von quantitativen und qualitativen Daten vorgenommen. Für die Datenerhebung wird zunächst in Punkt 7.1 die Ziele und der Aufbau erläutert. Danach werden die Erhebung und alle damit zusammenhängenden Planungs- und Vorbereitungsschritte durchgeführt und beschrieben. Im Anschluss an die Testreihen werden die gewonnen Daten mit Hilfe der statistischen Methoden normalisiert und ausgewertet beziehungsweise Interpretiert und mit den zu Anfang aufgestellten Behauptungen verglichen.Als letztes werden den selbst erhobenen Daten noch Sekundärversuche die schon bereits in der Vergangenheit von anderen durchgeführt wurden mit meinen Ergebnissen auf Unterschiede und mögliche Auslöser für diese analysiert.

Den letzten Abschnitt macht der Punkt \glqq Ausblick und Würdigung\glqq{} bei dem sich noch einmal die zuvor erarbeiteten und umgesetzten Methoden und Komponenten betrachtet werden und auf den zukünftigen Gebrauch analysiert werden. Am Ende der Ausarbeitung möchte ich persönlich noch einmal eine Würdigung an bestimmte Personen aussprechen die mir diese Ausarbeitung ermöglicht haben.

Während meines Vorgehens werde ich immer wieder versuchen Grafiken, Bilder und Tabellen einzubauen, die dabei helfen sollen dem Leser angewendete Methodiken und Vorgehensweisen verständlicher zu machen.




