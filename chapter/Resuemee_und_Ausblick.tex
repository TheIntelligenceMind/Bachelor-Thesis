\section{Resümee und Ausblick}
Als Abschluss des Projekts soll über den Verlauf und die Ergebnisse reflektiert werden. Darüber hinaus wird ein Ausblick zu möglichen Maßnahmen, die im Anschluss an das Projekt durchgeführt werden können, gegeben.

% Verlauf - Vorgehen in der Praxis rentabel?
Insgesamt kann das Projekt als erfolgreich abgeschlossen bezeichnet werden. Das primäre Projektziel wurde vollständig erreicht. Zudem hat das Projekt mit dem UIDataCollector ein Nebenprodukt abgeworfen, das es ermöglicht zukünftig strukturierte Aufzeichnungen von Interaktionen durchzuführen.

Mit Blick auf den Projektaufwand, ist die Rentabilität jedoch kritisch zu betrachten. Grundsätzlich scheint ein agiles Vorgehen in der Software-Entwicklung, besonders im Bereich von Oberflächen, ein sinnvoller Ansatz zu sein. Der Entwickler bekommt im Gegensatz zu klassischen Modellen einen viel präziseren Blick für die Anforderungen und die Anwender, für die die Schnittstelle konzipiert wird. Auf der anderen Seite hat sich herausgestellt, dass ein derartiger Aufwand für die Evaluation als \enquote{Ein-Mann-Projekt} schwierig zu bewältigen ist. Daher ist es in der Praxis empfehlenswert entweder eine abgespeckte Form der Evaluation durchzuführen oder in einem kleinen Team, das sich die Aufgaben untereinander aufteilen kann, zu arbeiten. 

Bei der Evaluation einer Schnittstelle ist sowohl die Methode schriftliche Befragung via Fragebogen als auch die Logfile-Erhebung zu empfehlen. Mit den Ergebnissen der Isonorm Fragebögen, können viele Informationen über die Meinung und Zufriedenheit der Anwender gewonnen werden. Durch das eigens entwickelte Tool, zur Aufzeichnung von Interaktionen, konnten sehr generische Daten erhoben werden. Aus diesen ließen sich eine Vielzahl von Erkenntnissen gewinnen, die für die Bewertung der Oberflächen genutzt werden konnten. Der Vorteil an der eigenen Entwicklung ist, dass von Projekt zu Projekt entschieden werden kann in welchem Detailgrad die Daten erhoben werden sollen. Umso generischer die Daten gesammelt werden, desto höher ist der Aufwand für die anschließende Aggregation und Auswertung einzuschätzen.

% gesparte Bearbeitungszeit pro Jahr
Rückblickend soll auch der Nutzen des Projekts etwas genauer verdeutlicht werden. Dazu wird, anhand der Ergebnisse in Kapitel \ref{sec:auswertungDerLogfiles}, die zeitliche Einsparung für die Bearbeitung eines Vorgangs auf ein Jahr hochgerechnet. Es wird von 68 Bearbeitungsvorgängen pro 80 Sachbearbeiter und pro Tag ausgegangen. Zudem wird von den ermittelten 14 Sekunden Einsparung pro Bearbeitungsvorgang ausgegangen. Wird nun auf 230 Arbeitstage im Jahr und insgesamt 550 Sachbearbeiter hochgerechnet. Es stellt sich heraus, dass 52 Arbeitstage à 8 Stunden pro Jahr mit der neuen Benutzeroberfläche eingespart werden könnten. Die Kennzahl gibt keine vollständige Sicherheit. Jedoch sollte sie ein Indiz für die Wirkungsrichtung des Projekts aufzeigen können.

Als Fazit lässt sich schließen, dass es sich lohnen kann Benutzeroberflächen zu erneuern bzw. zu überarbeiten. Besonders Oberflächen, die mit der Zeit an Komplexität dazugewonnen haben und dadurch unübersichtlich und ineffizient wurden, sind Kandidaten für eine Neuauflage. Dabei kann es durchaus sinnvoll sein mit Hilfe von Evaluationsmethoden eine bestehende Benutzerschnittstelle auf Schwächen und Stärken zu analysieren, um gegeben falls eine neue Oberfläche zu entwerfen. Ebenfalls können die Methoden dazu eingesetzt werden, den Unterschied von Oberflächen bewerten zu können. Bei der Konzeption hat es sich ebenfalls als hilfreich erwiesen, ein agiles Vorgehen zu verfolgen. Dieses besitzt seine Stärken in der inkrementellen und Nutzungskontext nahen Entwicklung 

% Ausblick
In der kurz- bis mittelfristigen Zukunft, ist es angedacht mit dem neu entwickelten Dialog weiterzuarbeiten, um bestenfalls den alten Dialog komplett ablösen zu können. Wie bereits angesprochen ist es auch denkbar, dass Logfile-Erhebungen weiterhin zum Einsatz kommen, um Oberflächen und deren Nutzungskontext genauer bewerten zu können. Mit Hilfe der gewonnen Erfahrungen und Erkenntnisse aus diesem Projekt, können zukünftig weitere Oberflächen überarbeitet werden. Langfristig sollte es das Ziel sein, dass alle Oberflächen die umliegenden Bearbeitungsprozesse des Inkassoverfahrens optimal unterstützen.