\section{Die neue grafische Benutzerschnittstelle}
Eine Hauptaufgabe des Projektes ist der Entwurf und die Implementierung einer neuen grafischen Benutzerschnittstelle für die Eingabe von Vermögensverzeichnissen. Diese Schnittstelle soll als Erweiterung in die bestehende Software COSIMA eingebettet werden. Die neue Eingabemaske soll sich an den Gestaltungsgesetzen, Normen und Usability Heuristiken orientieren und die darin enthaltenen Empfehlungen und Prinzipien berücksichtigen.
In den folgenden Unterabschnitten wird insbesondere auf die Anforderungsdefinition und die konzeptionellen Hintergründe der neuen Benutzerschnittstellen eingegangen.

\subsection{Anforderungsdefinition}
Für die initiale Projektvorstellung meiner Bachelor Thesis, wurden im Vorhinein bereits mehrere fachlich und technisch visierte Ansprechpartner mit einbezogen. Es wurden mehrere Gespräche sowohl mit den fachlichen Anforderungsstellern als auch mit den technischen Beteiligten geführt, um die Anforderungen für den neuen Dialog möglichst genau zu definieren. Die folgende Liste, mit funktionalen und nicht funktionalen Anforderungen, ist das Resultat der Anforderungsgespräche:
\begin{itemize}
	\item 
\end{itemize}


\subsection{Konzeption und Design}

\subsection{Implementierung und Einführung}
