\section{Evaluation}
Als nächstes soll die Methodik der Evaluation (siehe Kap. 4.3) angewandt werden. Dazu sollen zunächst die Ziele und der Aufbau der Evaluation definiert werden. Da es sich hierbei um eine empirische Evaluation handelt, werden Daten mittels geeigneter Methoden erhoben. Diese Methoden werden nachfolgend ausgewählt und genauer beschrieben.

%%%%%%%%%%%%%%%%%%%%%%%%%%%%%%%%%%%%%%%%%%%%%%%%%%%%%%%%%%%%%%%%%%%%%%%%%%%%

\subsection{Ziel und Aufbau}
Ziel dieser Evaluation ist es, zwei Benutzerschnittstellen hinsichtlich ihrer Effizienz und Ergonomie zu vergleichen. Dabei sind der neu entworfene Dialog aus dem vorherigen Abschnitt \enquote{Erarbeitung der Gestaltungslösung} (siehe Abb. \ref{fig:neuerDialog}), der bereits konzipiert und implementiert wurde und der aktuell im produktiven Einsatz genutzte Dialog für Vermögensverzeichnisse (siehe Abb. \ref{fig:aktuellerDialog}) Gegenstand des Vergleichs. Das Ziel der Evaluation ist bewusst an die Forschungsfrage angelehnt, denn die Evaluation soll später zur Beantwortung bzw. Überprüfung der Forschungsfrage sowie der Hypothesen benutzt werden.

Der Aufbau der Evaluation gliedert sich in vier Phasen. Die erste Phase beschäftigt sich mit generellen Planungstätigkeiten, zu denen auch die Definition des Ziels sowie der Untersuchungsgegenstände zählt. In der darauf folgenden Phase wird die Stichprobe und die Erhebungsmethoden und Messinstrumente für die Erhebung ausgewählt. Anschließend werden in der dritten Phase die Erhebungsmethoden auf die Stichprobe angewandt. In der letzten Phase werden dann in Kapitel \ref{sec:ergebnisseUndDiskussion} alle Datensätze und Erkenntnisse ausgewertet. Zusätzlich wird in Kapitel \ref{sec:KNAZusaetzlichesKriterium} eine minimale Form der \gls{KNA} durchgeführt, um die Kosten und Nutzen des Projektes grob aufzuzeigen. Schließlich wird mit Hilfe der Evaluationsergebnisse die Hypothesen überprüft und die Forschungsfrage beantwortet.

%%%%%%%%%%%%%%%%%%%%%%%%%%%%%%%%%%%%%%%%%%%%%%%%%%%%%%%%%%%%%%%%%%%%%%%%%%%%

\subsection{Empirische Erhebung}
\label{sec:empirischeErhebung}
Zunächst sollen geeignet Verfahren und eine repräsentative Stichprobe für die Datenerhebung bestimmt werden. Diese werden dann im zweiten Schritt angewandt.

%%%%%%%%%%%%%%%%%%%%%%%%%%%%%%%%%%%%%%%%%%%%%%%%%%%%%%%%%%%%%%%%%%%%%%%%%%%%

\subsubsection{Auswahl der Stichprobe}
Damit im nächsten Schritt eine Auswahl der Stichprobe geschehen kann, muss zu aller erst definiert werden welche und bestenfalls wie viele Personen die Gesamtheit darstellen. Da es sich bei der Evaluation, um die Bewertung der Benutzerschnittstelle für Vermögensauskünfte handelt, zählen alle Personen zur Gesamtheit, dessen tägliche Arbeit mit dem Anwendungsfall \enquote{Eingabe eines Vermögensverzeichnisses} zu tun hat. Das heißt jeder Sachbearbeiter der getrieben durch seine Aufgaben, die Oberfläche für Vermögensauskünfte, zur Eingabe eines neuen Vermögensverzeichnisses benutzt ist potentieller Teil der Grundgesamtheit. In Zahlen ausgedrückt sind dies mehrere hundert Personen. Genaue Zahlen sind sehr schwierig zu ermitteln, daher wird für nachfolgende Aussagen und Rechnungen von einer gerundeten Grundgesamtheit von etwa ??? Personen ausgegangen.

Bei der Auswahl der Stichprobe wird auf eine Kombination der bewussten Auswahl und der Zufallsziehung gesetzt. Es liegt keine vollständig zufällige Stichprobe vor, da bereits ein Benutzerprofil in Kapitel \ref{sec:benutzergruppen} erstellt wurde und dies als Wissensgrundlage durchaus genutzt werden kann. Es wird eine mehrstufige Zufallsziehung durchgeführt. Diese hat den Vorteil, dass man sicher aus jeder Kategorie bzw. Abteilung einen Teil der Stichprobe zieht und nicht wie bei einer reinen Zufallsziehung eventuell nur aus einer Abteilung Probanden bekommt. Dadurch erhöht sich die Repräsentativität der Stichprobe. Die Gruppen aus denen die Teilnehmer gezogen werden sind die Abteilungen in denen die schriftliche Sachbearbeitung den oben genannten Anwendungsfall bearbeiten. Die Stichprobe setzt sich wie folgt zusammen:
\begin{itemize}
    \item Es sind insgesamt 83 Probanden.
    \item Die Probanden sind in absoluten Zahlen nicht gleichstark aus jeder Abteilung vertreten, da die Team bzw. Abteilungsgrößen variieren. Es wird aber davon ausgegangen, dass relativ gesehen jede Abteilungen in die gleiche Gewichtung in der Stichprobe aus macht.
    \item Aufgrund der Urlaubsplanungen, ist mit Schwankungen in der Anwesenheit zu rechnen, daher werden nie alle 83 Sachbearbeiter zu einer Zeit präsent sein.
\end{itemize}

%%%%%%%%%%%%%%%%%%%%%%%%%%%%%%%%%%%%%%%%%%%%%%%%%%%%%%%%%%%%%%%%%%%%%%%%%%%%

\subsubsection{Auswahl der Erhebungsmethoden und Messinstrumente}
Im folgenden werden die Methoden der Logfile-Erhebung und der Befragung via Fragebogen auf den Kontext angewandt werden. Dazu gehört, dass die benötigten Messinstrumente genauer erklärt und vorgestellt werden. Bei der Auswahl der Erhebungsmethoden und Messinstrumente wurde bewusst auf das Zusammenspiel zweier Methoden gesetzt. Die Stärken und Schwächen der Methoden ergänzen bzw. erweitern sich gegenseitig\footnote{\cite[vgl.][5\psq]{Priemer2004}}. Ziel ist es, eine möglichst hohe Informationsdichte, Genauigkeit und Aussagekraft der Ergebnisse zu erreichen. 

\textbf{Logfile-Erhebung}

Als erstes und gleichzeitig als zuerst durchgeführtes Verfahren wird die Logfile-Erhebung (siehe Kap. \ref{sec:erhebungsmethoden}) gewählt. Diese eignet sich besonders gut, da viele Daten über das Verhalten der Benutzer im jeweiligen Dialog gesammelt werden können. Die Methodik wird besonders mit dem Ziel der Zeitmessung eingesetzt. Zusätzlich sollen Informationen über Tastatur und Maus Interaktionen aufgezeichnet werden, damit später ein Vergleich dieser beiden Interaktionsarten stattfinden kann. Da der Blickpunkt der Evaluation auf der Ergonomie und Effizienz der Dialoge liegt, sind Zeiten und das Interaktionsverhalten gute Referenzwerte, um Aussagen zu diesen zwei Bewertungskriterien treffen zu können.

Für die Logfile-Erhebung gilt es eine Möglichkeit zu schaffen, mit der Daten, die bei Mensch-System-Interaktionen entstehen, automatisch zu sammeln. Es gibt auch Standardsoftware, jedoch ist diese überwiegend für die Web-Umgebung konzipiert oder kann diesen speziellen Fall nicht wirkliche abdecken. Daher wurde speziell ein Customized Logging Tool, also ein Programm das clientseitig explizit die Nutzung einer Software aufzeichnet, entwickelt\footnote{\cite[vgl.][3]{Priemer2004}}. Das Tool trägt den internen Namen UIDataCollector (im Folgenden nur noch UIDC). Der UIDC wurde für das Sammeln und Speichern von Bewegungsdaten bzw. Interaktionsdaten, in der Cosima Oberfläche, konzipiert. Er soll die Grundlage für eine objektive Datenerhebung zur Verfügung stellen.

Im Folgenden soll nur auf ein paar Eckpunkte bzw. Anforderungen eingegangen werden, die bei der Entwicklung des Tools berücksichtigt wurden.
\begin{compactitem}
   \item Es wird bei jeder Interaktion, die ein Anwender auslöst, ein Datensatz erzeugt, der in einer relationalen Datenbank, in Form von relational abhängigen Datenobjekten, gespeichert wird.
   \item Eine Interaktion kann sein: das Klicken einer Schaltfläche, das Wechseln eines Bedienelements, das Heraus- oder Hereinspringen in den Dialog.
   \item Das Sammeln von Interaktionen kann auf eine gewisse Benutzergruppe und auf gewisse Dialoge eingeschränkt werden.
\end{compactitem}

Die Datenhaltung für die Interaktionsdaten basiert auf zwei Datenobjekten in denen alle nötigen Informationen gespeichert werden. Es gibt Session-Objekte und Interaktions-Objekte. Ein Session-Objekt beschreibt genau eine Session und dessen Interaktionen. Eine Session startet beim Öffnen eines Dialoges und endet beim Schließen des selben Dialoges. Dabei können innerhalb einer Session mehrere Interaktionen stattfinden. Eine Interaktion ist immer klar einer Session zugeordnet und bekommt über diese eine Eindeutigkeit. Zudem kann eine Interaktion immer nur zusammen mit einer Session aber nicht alleine existieren. Die Datenhaltungsobjekte mit ihren Attributen und Beziehungen zueinander werden als \gls{ERM} im Anhang \ref{sec:ermUIDataCollector} veranschaulicht.

Die Daten aus der Logfile-Erhebung werden im Kapitel \ref{sec:auswertungDerLogfiles} ausgewertet und in einen Zusammenhang mit den Fragebögen Ergebnissen (siehe Kap. \ref{sec:auswertungDerFrageboegen}) gebracht.

\textbf{Befragung via Fragebogen}

Die zweite Erhebung soll durch eine Befragung via standardisierten Fragebogen erfolgen. Der zum Einsatz kommende standardisierte ISO 9241-10 Fragebogen wurde bereits in Kapitel \ref{sec:erhebungsmethoden} vorgestellt. Der Fragebogen dient als abschließendes Resümee. Hierbei sollen die Benutzer ihre persönlichen Eindrücke und Erfahrungen zur Handhabung mit dem alten als auch mit dem neuen Vermögensverzeichnis Dialog äußern.

%%%%%%%%%%%%%%%%%%%%%%%%%%%%%%%%%%%%%%%%%%%%%%%%%%%%%%%%%%%%%%%%%%%%%%%%%%%%

\subsubsection{Durchführung}
\label{sec:durchfuehrungEvaluation}
Die Durchführung der beiden Erhebungsmethoden soll sequentiell erfolgen. Das heißt, dass erst die Logfile-Erhebung durchgeführt wird und anschließend Fragebögen ausgefüllt werden. Die Logfile-Daten zum neuen Dialog werden über einen Zeitraum von zwei Wochen gesammelt. In der Zeit sollen die Teilnehmer im Produktivbetrieb ausschließlich mit dem neuen Dialog arbeiten. Damit der neue Dialog mit dem alten Dialog verglichen werden kann wurden zuvor bereits zwei Wochen lang Logfile-Daten zum alten Dialog gespeichert. 

Vorab wurde den Teilnehmern eine ausführliche Anleitung mit Bildern und Funktionen der neuen Benutzerschnittstelle zugeschickt. Diese Anleitung sollte den Anwendern die Möglichkeit geben erste Eindrücke zu gewinnen und vor Beginn des Testzeitraums Fragen zu stellen, damit während der Testphase möglichst unbeschwert mit dem neuen Dialog gearbeitet werden kann. Zudem wurden die Probanden im Vorhinein zu Datenschutzaspekten aufgeklärt. Mit einer anschließenden Einwilligungsbestätigung konnten die Probanden diesen zustimmen oder ablehnen und somit die Teilnahme verweigern. Alle Probanden waren einverstanden und haben sich bereit erklärt teilzunehmen.

Im Anschluss an die Aufzeichnung von Interaktionsdaten der alten sowie neuen Benutzerschnittstelle, bekommen die Teilnehmer den ISO 9241-110 Fragebogen ausgehändigt und sollen diesen ausfüllen. Es wurde beim Verteilen der Fragebögen drauf hingewiesen, dass nur die Personen die auch den neuen Dialog produktiv testen konnten den Fragebogen ausfüllen sollen. Es sollen falsche Einschätzungen ausgeschlossen werden. Bei der Erstellung und Verteilung der Fragebögen wurden mehrere Aspekte beachtet, um eine höhere Rücklaufquoute zu erzielen. Zum einen wurde im Anschreiben darauf geachtet, dass die Teilnehmer direkt angesprochen werden und es wurde auf den Stellenwert des Projekts und dessen Effekte hingewiesen. Eine weitere Maßnahme während des Befragungszeitraums war es, die Befragten auf die Befragung und den Ablauf der Frist hinzuweisen\footnote{\cite[vgl.][58]{Petermann2005}}. Von den Befragten haben 48 innerhalb der Frist abgegeben. Der Rest hat entweder, aufgrund von fehlender Testzeit bewusst, unbewusst oder aus Urlaubs- bzw. Krankheitsgründen nicht teilgenommen. Damit ergibt sich eine Rücklaufqoute von etwa 58\%. Diese liegt laut Literaturangaben in einem durchschnittlichen und akzeptablen Rahmen für schriftliche Befragungen\footnote{\cite[vgl.][58]{Petermann2005}}. Die Fragebögen wurden nach Ablauf der Frist eingesammelt und werden ebenso wie die Logfile-Daten im Kapitel \ref{sec:auswertungDerFrageboegen} ausgewertet.
